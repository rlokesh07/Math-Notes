\documentclass[../main.tex]{subfiles}

\usepackage{amsmath}
\usepackage{xcolor}
\begin{document}


\section{Parametric Equations and Polar Coordinates}

\subsection{Parametric Equations}

If x and y are continuous functions, then 
\[ x = x(t) \text{ and } y = y(t) \]
are parametric equations, and t is the parameter. The points that are obtained 
when t is varied over an interval is the graph, or is called the parametric curve.
\smallskip
\par To understand parametric functions further, you can eliminate the parameter by 
relating t to x and y. For the function $x(t) = 4\cos t$ and $y(t) = 3\sin t$, 
divide both sides by 4 and 3 respectivly and plug the cosine and sin functions into 
the Pythagoreus identity to create the equation of a circle.

\subsection{Calculus on Parametric Curves}
\subsubsection{Derivatives of Parametric Equations}
Given a curve defined by $x = x(t)$ and $y = y(t)$, given that their derivatives 
exist, the derivative is given by

\[ \frac{y'(t)}{x'(t)} \]

The second derivative is given by 

\[ \frac{\frac{d}{dt}(\frac{dy}{dx})}{\frac{dx}{dt}} \]

by using the quotient rule on the first derivative

\subsubsection{Integrals of Parametric Equations}

The area under a parametric curve is

\[ \int_a^b y(t)x'(t) dt \]

\subsubsection{Arc Length of a Parametric Curve}

The length of a arc on a parametric curve is

\[ s = \int_{t_1}^{t_2} \sqrt{(\frac{dx}{dt})^2 + (\frac{dy}{dt})^2} \quad  dt \] 

\subsubsection{Surface Area of a Parametric Curve}

When a parametric curve is rotated about the x axis, the area is, 
\[ S = 2 \pi \int_a^b f(x) \sqrt{1+(f'(x))^2} \quad dx \]
\end{document}
