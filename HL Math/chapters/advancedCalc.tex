\documentclass[../main.tex]{subfiles}

\usepackage{amsmath}

\begin{document}


\section{Advanced Calc}


\subsection{L'hopital rule}

If a limit gives you a indeterminate form, ie. $\frac{\inf}{\inf}$ or $\frac{0}{0}$, you can use the derivatve of the numerator and denominator to try to find the limit.

\[ \lim_{x \to c} \frac{f(x)}{g(x)} = \lim_{x \to c} \frac{f'(x)}{g'(x)} \]

\subsection{Seperable Differential Equations}

A seperable differential equation is any equation that can be written in the form $y' = f(x)g(x)$. This means that you can seperate the RHS into two seperate functions.

First, check for values of y that make g(y) = 0, these lead to constant solutions. Then, you rewrite the euqation to the form of


\[ \frac{dy}{g(y)} = f(x)dx \]

and you integrate both sides. Because both sides have c, there is only one. Solve the equation for y and if there is an intial condition sub in the value and solve for c.

\subsection{Homogeneous Differential Equations}

\[ 2x dy = (x+y)dx \]

Notice how all the terms have the same order. Sub in $y$ as $vx$. That means that $dy = xdv + vdx$ through the power rule.


\[ 2x(xdv + vdx) = (x+ vx)dx \]

You can factor out an x on the right side


\[ 2x(xdv + vdx) = x(1 + v)dx \]

Cancel out the x on both sides

\[ 2(xdv + vdx) = (1 + v)dx \]


\[ 2xdv + 2vdx = (1 + v)dx \]

Then, consolidating the vdx on both sides

\[ 2xdv + vdx = dx \]

Then, movign the vdx to the other side and factoring out dx again


\[ 2xdv= dx - vdx \]

\[ 2xdv= dx(1 - v) \]

Now, isolating x and v, we move some terms around


\[ \frac{dv}{(1-v)} = \frac{dx}{2x} \]

We can finally integrate both sides


\[ \int{\frac{dv}{(1-v)}} = \int{\frac{dx}{2x}} \]

\[ -ln(1-v) = 0.5 ln(x) + c \]

You can use log rules and stuff to go beyond that, but when it is time you can sub v back out for $\frac{y}{x}$ 


\end{document}
