\documentclass[../main.tex]{subfiles}

\usepackage{amsmath}

\begin{document}


\section{Trig Functions}

There are 6 trig functions in the curriculum,

\[ \sin \]
\[ \cos \]
\[ \tan \]

and their reciprocols,

\[ \csc \]
\[ \sec \]
\[ \cot \]

relating the ratios of the sides with the trig functions to the Pythagorean Theorum gives way to the Pythagorean identities

\[ 1 + \tan^2 \theta  = \sec^2 \theta \]
\[ 1 + \cot^2 \theta = \csc^2 \theta \]

\section{Inverse Trig Functions}

Note that $\sin^-1$ is also known as $arcsin$, which is the inverse of sin. Plugging in the sin value will give you the angle. Same goes for $arccos$ and $arctan$ giving the angles for $\cos$ and $\tan$. 

In order for these arc functions to be functions, they have a range restriction. $arcsin$ has a range of $-\frac{\pi}{2} \leq y \leq \frac{\pi}{2}$, $arccos$ has a range of $0 \leq y \leq \pi$, and $arctan$ has a range of $-\frac{\pi}{2} \leq y \leq \frac{\pi}{2}$ 

\section{Compound Angle Identities}

\subsection{Compound Identities}

\[ \sin(A \pm B) = \sin(A) \cos(B) \pm \cos(A) \sin(B) \]
\[ \cos(A \pm B) = \cos(A) \cos(B) \mp \sin(A) \sin(B) \]

\[ \tan(A \pm B) = \frac{\tan A \pm \tan B}{1 \mp \tan(A) \tan(B)} \]


\subsection{Cofunction Identities}

\[ \sin(\frac{\pi}{2} - x) = \cos x \]
\[ \cos(\frac{\pi}{2} - x) = \sin x \]
\[ \csc(\frac{\pi}{2} - x) = \sec x \]
\[ \sec(\frac{\pi}{2} - x) = \csc x \]
\[ \tan(\frac{\pi}{2} - x) = \cot x \]
\[ \cot(\frac{\pi}{2} - x) = \tan x \]


\subsection{Examples}

Let $f(x) = \tan(x + \pi) \cos(x - \frac{\pi}{2}) where $0 < x < \frac{pi}{2}$. Express $f(x)$ in terms of $\sin x$ and $\cos x$

\begin{align*}

    f(x) &= \tan(x + \pi) \cos(x - \frac{\pi}{2}) \\

         &= \tan(x) \cos(x - \frac{\pi}{2}) \\

         &= \tan(x) \sin(x) \\

         &= \frac{\sin^2 x}{\cos x}

\end{align*}


\end{document}
