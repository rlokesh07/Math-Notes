\documentclass[../main.tex]{subfiles}



\begin{document}


\section{Trig Functions}

There are 6 trig functions in the curriculum,

\[ \sin \]
\[ \cos \]
\[ \tan \]

and their reciprocols,

\[ \csc \]
\[ \sec \]
\[ \cot \]

relating the ratios of the sides with the trig functions to the Pythagorean Theorum gives way to the Pythagorean identities

\[ 1 + \tan^2 \theta  = \sec^2 \theta \]
\[ 1 + \cot^2 \theta = \csc^2 \theta \]

\section{Inverse Trig Functions}

Note that $\sin^-1$ is also known as $arcsin$, which is the inverse of sin. Plugging in the sin value will give you the angle. Same goes for $arccos$ and $arctan$ giving the angles for $\cos$ and $\tan$. 

In order for these arc functions to be functions, they have a range restriction. $arcsin$ has a range of $-\frac{\pi}{2} \leq y \leq \frac{\pi}{2}$, $arccos$ has a range of $0 \leq y \leq \pi$, and $arctan$ has a range of $-\frac{\pi}{2} \leq y \leq \frac{\pi}{2}$ 

\section{Compound Angle Identities}




\end{document}
